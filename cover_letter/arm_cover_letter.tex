%Author by Rajib Das Bhagat (rajibdasbhagat@gmail.com)
%
\documentclass[11pt,a4paper,roman]{moderncv}
\usepackage[english]{babel}

\moderncvstyle{classic}
\moderncvcolor{black}

% character encoding
\usepackage[utf8]{inputenc}

% adjust the page margins
\usepackage[scale=0.80]{geometry}

% personal data
\name{Dinesh}{Adepu}
\email{adepu.dinesh.a@gmail.com}
\phone[mobile]{+91 9963943672}
\address{11-29-106, 2nd Bank Colony, Warangal, India}


\begin{document}

\recipient{To}{The Project Incharge, \\
               Arm,\\
               Manchester,\\
               United Kingdom}
\date{\today}
\opening{\textbf{Sub: Staff Software Engineer - Numerical Software.}}
% with reference advertisement: XXXX/XX, dated 04-03-2022
\closing{Your Sincerly, \vspace{-1em}}


% \enclosure[Enclosed (Marksheets/Certificates)]{
%               \\ 1. SSLC (Admit/MarkSheet/Certificate)
%                  \hspace{0.0em} 5. GATE (2016) \hspace{2em} 9. Identity-Proof (Voter-Id)\\
%                  2. Diploma (MarkSheet/Certificate)
%                  \hspace{1.8em} 6. SLET (2021) \hspace{1.9em} 10. Resume\\
%                  3. B.Tech (MarkSheet/Certificate)
%                  \hspace{2.3em} 7. UGC-NET (2022) \hspace{-0.1em} \\
%                  4. M.Tech (MarkSheet/Certificate)
%                  \hspace{2.1em} 8. Caste Certificate \\
%                  }
\makelettertitle



Dear team,
\\
%references such as what and how you got this information
\vspace{1em}
I am Dinesh Adepu, and I hold an M.Tech and Ph.D. in Aerospace
Engineering from IIT Bombay, followed by a postdoctoral fellowship in
the Chemical Engineering Department at the University of
Surrey. Currently, I am working as a postdoctoral researcher at IIT
Delhi, India. \\


\vspace{1em}
During my Ph.D., I developed a novel numerical technique in Smoothed
Particle Hydrodynamics (SPH) to study fluid and elastic dynamics. This
method was further enhanced by coupling it with the Discrete Element
Method (DEM) to solve rigid-fluid coupling problems, incorporating a
unique contact force model for simulating interactions between
arbitrarily shaped bodies. The technique was applied in several
multiphysics scenarios, such as fluid-structure interaction,
rigid-fluid coupling, and solid-body erosion. All of this work was
implemented in PySPH and is fully open-source, ensuring
reproducibility.\\


%narrate why you want to apply
\vspace{1em}
In addition, I developed parallel n-body software using the
memory-safe Rust programming language, where I implemented DEM within
this framework. I also have substantial experience in object-oriented
programming and have worked extensively with Linux operating systems
throughout my academic career.\\


\vspace{1em}
As a postdoctoral researcher at the University of Surrey, I
contributed to an EPSRC-funded project studying particle dispersion,
swelling, and agglomeration in stirring tanks using an SPH-DEM
solver. This project led to the development of an open-source solver
integrated with PySPH and DEM. To address large-scale mixing
challenges, I also created an exascale-capable DEM solver using the
Cabana framework, built on C++ and leveraging Kokkos for heterogeneous
computing.\\



\vspace{1em}
I am excited by the prospect of delivering high-performance solutions
to end users by developing efficient, low-level numerical code. With
extensive experience in C++, C, and Fortran, I have honed my
programming skills through developing and optimizing code on Linux,
utilizing git for version control, and participating in code reviews
to ensure best practices. I thrive in collaborative environments and
look forward to contributing to a team of software engineers, where I
can share innovative ideas and provide constructive feedback to
enhance the quality and performance of the software.\\



\vspace{1em}
Thank you for considering my application. I look forward to the
opportunity to discuss my qualifications further.\\

% Sincerely,


\vspace{0.5cm}


\makeletterclosing

\end{document}
