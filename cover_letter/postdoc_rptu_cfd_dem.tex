%Author by Rajib Das Bhagat (rajibdasbhagat@gmail.com)
%
\documentclass[11pt,a4paper,roman]{moderncv}
\usepackage[english]{babel}

\moderncvstyle{classic}
\moderncvcolor{black}

% character encoding
\usepackage[utf8]{inputenc}

% adjust the page margins
\usepackage[scale=0.80]{geometry}

% personal data
\name{Dinesh}{Adepu}
\email{adepu.dinesh.a@gmail.com}
\phone[mobile]{+91 9963943672}
\address{11-29-106, 2nd Bank Colony, Warangal, India}


\begin{document}

\recipient{To}{Prof. Sergiy Antonyuk, \\
               RPTU,\\
               Germany}
\date{\today}
\opening{\textbf{Sub: Application for the Postdoc position: “Advanced Particle Technology for Wastewater Phosphorus Recovery“.}}
% with reference advertisement: XXXX/XX, dated 04-03-2022
\closing{Your Sincerly, \vspace{-1em}}


% \enclosure[Enclosed (Marksheets/Certificates)]{
%               \\ 1. SSLC (Admit/MarkSheet/Certificate)
%                  \hspace{0.0em} 5. GATE (2016) \hspace{2em} 9. Identity-Proof (Voter-Id)\\
%                  2. Diploma (MarkSheet/Certificate)
%                  \hspace{1.8em} 6. SLET (2021) \hspace{1.9em} 10. Resume\\
%                  3. B.Tech (MarkSheet/Certificate)
%                  \hspace{2.3em} 7. UGC-NET (2022) \hspace{-0.1em} \\
%                  4. M.Tech (MarkSheet/Certificate)
%                  \hspace{2.1em} 8. Caste Certificate \\
%                  }
\makelettertitle



Dear Prof. Sergiy Antonyuk,
\\
%references such as what and how you got this information
\vspace{1em}
I am Dinesh Adepu, and I hold an M.Tech and Ph.D. in Aerospace
Engineering from IIT Bombay, followed by a postdoctoral fellowship in
the Chemical Engineering Department at the University of
Surrey. Currently, I am working as a postdoctoral researcher at IIT
Delhi, India. \\


\vspace{1em}
During my Ph.D., I developed a novel numerical technique in Smoothed
Particle Hydrodynamics (SPH) to study fluid and elastic dynamics.
Further, the developed method is coupled with the discrete element
method (DEM) to handle the rigid-fluid coupling, where ocean
engineering and process engineering problems are handled.  Specifically, a new contact force
model is used to model the interaction between arbitrarily shaped
bodies. As a multiphysics application, the developed technique is
applied to model the fluid-structure interaction phenomenon. All the
above work is implemented in PySPH and is fully open-source and
reproducible. I have developed a parallel n-body software in
memory-safe Rust programming language, in which DEM is implemented. \\


\vspace{1em}
As a postdoctoral researcher at the University of Surrey under Prof. Chuan-Yu Wu, I
contributed to an EPSRC-funded project studying particle dispersion,
swelling, and agglomeration in stirring tanks using an SPH-DEM
solver. This project led to the development of an open-source solver
integrated with PySPH and DEM. To address large-scale mixing
challenges, I also created an exascale-capable DEM solver using the
Cabana framework, built on C++ and leveraging Kokkos for heterogeneous
computing.\\


\vspace{1em}
With my demonstrated experience in addressing multiphase
problems—particularly through the application of numerical techniques
such as Smoothed Particle Hydrodynamics (SPH) and the Discrete Element
Method (DEM)—I believe I am well-qualified for the proposed
position. I am confident that my proven ability to tackle complex
particle dispersion and multiphase flow challenges using these methods
will bring significant value to the institution and contribute
meaningfully to the success of the project and the company.\\



\vspace{1em}
Thank you for considering my application. I look forward to the
opportunity to discuss my qualifications further.\\

% Sincerely,


\vspace{0.5cm}


\makeletterclosing

\end{document}
